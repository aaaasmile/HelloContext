% language=it
\setupcolors[state=start]
\definecolor[headingcolor][r=0.1,g=0.6, b=0.2]

\setupbackgrounds
	[text]
	[background=color,
	backgroundcolor=color-1]
\setuphead[section,chapter,subject][color=headingcolor]
\definehighlight
[imp] %nota che questo è l'ancora per avere questa definizione di highlight. Può venire usato come \imp{testo in evidenza}
[color=red]


\starttext
\subject{Questo è un titolo meraviglioso}
Ciao
\samplefile{tufte}
% \samplefile{klein}

\color[red]{This will show up red.} \\
\colored[r=0.5]{also red} \\
Se c'è un testo che è \highlight[imp]{testo molto importante} viene mostrato \imp{naturalmente} come importante.
% \showcolorcomponents[red,green,blue,yellow,magenta,cyan,black]

\definecolor[t:only] [a=1,t=.5]
\dontleavehmode
\blackrule[width=4cm,height=1cm,color=darkgreen]%
\hskip-2cm
\color[t:only]{\blackrule[width=4cm,height=1cm,color=darkred]}%
\hskip-2cm
\color[t:only]{\blackrule[width=4cm,height=1cm]}

\startMPcode
fill fullcircle scaled 3cm withcolor
.5 * spotcolor("whatever",(.3,.4,.5)) ;
fill fullcircle scaled 2cm withcolor
spotcolor("whatever",(.3,.4,.5)) ;
fill fullcircle scaled 1cm withcolor
spotcolor("whatever",(.3,.4,.5)/2) ;
\stopMPcode

\blackrule[color=MyColor,width=3cm,height=1cm,depth=0cm]

\definecolor[MyColor][r=.25,g=.50,b=.75]
\bgroup
	\definecolor[MyColor][s=.5]
	\startMPcode
		pickup pencircle scaled 4mm ;
		draw fullcircle scaled 30mm withcolor \MPcolor{MyColor} ;
		draw fullcircle scaled 15mm withcolor "MyColor" ;
	\stopMPcode
\egroup
\quad
\startMPcode
	pickup pencircle scaled 4mm ;
	draw fullcircle scaled 30mm withcolor \MPcolor{MyColor} ;
	draw fullcircle scaled 15mm withcolor "MyColor" ;
\stopMPcode


\startplacefigure[location={top,none}]
	[width=1cm,
	height=1cm,
	frame=off,
	background=headingcolor,
	backgroundcolor=headingcolor]
	{Un'immagine della vita}
\stopplacefigure
\stoptext

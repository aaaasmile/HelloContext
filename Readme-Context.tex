\setuplayout[topspace=0.5in, backspace=1in, header=24pt, footer=36pt,
  height=middle, width=middle]
% uncomment the next line to see the layout
%\showframe
\setuppagenumbering[location={header,right}]

\setupbodyfont[pagella]

\definetyping[powsh][bodyfont=small]
\definetyping[sh][bodyfont=small]

%urls
% nota come la tilde viene rappresentata
\useURL[Lpeg][http://www.inf.puc-rio.br/$\sim$roberto/lpeg][][Lpeg]

\starttext

\title{ConText}
Invece di mettere un {\em Readme} in formato .md questa volta uso  il formato .tex.
Così riesco a  creare un file Pdf e testare le funzioni di questo, per me  nuovo, pacchetto Tex.
\ConTeXt\ è un altro engine per creare Pdf partendo da un sorgente in Tex. È uno degli ultimi sviluppi per quanto
riguarda il mondo Tex e si basa su LuaTex. Il sito di riferimento è wiki.contextgarden.net.

\subject{Installazione}
L'installazione non la trovo immediata. C'è un Wiki su https://wiki.contextgarden.net/Installation
dove viene indicato un link per il sistema win64. Questo link, però, non riesco a scaricarlo in quanto lo zip sembra bloccato dal browser Brave.
Allora uso in un WLS (Window Linux System) il seguente comando:
\startsh
wget http://lmtx.pragma-ade.com/install-lmtx/context-win64.zip
\stopsh

Ho anche scaricato predentemente uno zip che, da quanto ho capito, mi sembra essere il sorgente di tutto il progetto, quindi niente di
eseguibile. Questo zip l'ho scaricato da http://www.pragma-ade.com/download-1.htm e l'ho scompattato in

\type{D:\scratch\latex\ConTex\source-distribution}

Questi I sorgenti non li ho utilizzati.

Lo zip context-win64.zip per windows l'ho scompattato in:

\type{D:\scratch\latex\ConTex}.
Nel file  context-win64.zip si trova un installer che non fa altro che scaricare tutti gli eseguibili mancanti per win64, assieme a tutti i vari font, e
li mette nella sottodirectory tex. L'eseguibile mtxrun.exe che crea il Pdf è ora in

\type{D:\scratch\latex\ConTex\tex\texmf-win64\bin}

Ho creato la directory {\bold _my-project} dove ho messo i miei primi esperimenti con ConTex come questo file e altri esempi
introduttivi presi dal wiki o dal file install.pdf, così come la patch di SciTe. La prima impressione non è affatto male e credo
che continuerò a provarlo. XeLatex (TinyTex) e sopratutto Latex (MikTex) non mi entusiasmano più di tanto.

\subject{Uso}
Per generare il file Readme-Context.pdf dal file Readme-Context.tex uso:
\startpowsh
context Readme-Context.tex
\stoppowsh
Nota bene che setpath.bat funziona solo quando uso cmd.exe.
In powershell, invece, il path l'ho settato con il seguente comando:
\startpowsh
$env:path = "D:\scratch\latex\ConTex\tex\texmf-win64\bin;" + $env:path
\stoppowsh


\subject{Scite}
Per vedere come alla fine ho usato SciTe guarda la \at{pagina}[lpegscite].

Il programma SciTe è molto semplice da installare (non ha l'installer) e si trova su: https://www.scintilla.org/SciTEDownload.html.
Ho scaricato la versione a 64 bit. Ho perso l'abitudine di mettere ogni programma che uso nel Path globale di windows, mentre preferisco
settarlo manualmente quando effettivamente lo uso.
Per lanciare SciTe  in combinazione con {\Context} uso la seguente command line:
\startpowsh
cd D:\scratch\latex\ConTex
setpath.bat
cd .\scite\bin
SciTe-WithLpeg.exe
\stoppowsh

Per l'installazione del lexer di  {\Context} in SciTe vado a copiare la directory
context in \type{D:\scratch\latex\ConTex\tex\texmf-context\context\data\scite} nella directory dove risiede SciTe.
Poi edito il file SciTEGlobal.properties aggiungendo la linea alla fine del file:

\type{import context/scite-context-user}

Per quanto riguarda il Lexer di {\Context},  usando SciTe scaricato nel formato binario dal sito ufficiale
non riesco a farlo andare in quanto il modulo di Lua lpeg non viene riconosciuto. Questo è dovuto al fatto che
il binario di SciTe distribuito dal sito Scintilla non contiene Lpeg e penso che difficilemnte lo conterrà nel futuro.

SciTe con {\Context} risulta impostato per usare il pdf viewer https://www.sumatrapdfreader.org/downloadafter che è un exe standalone.
L'ho messo nella stessa directory dove è {\Context}.
Così com'è, però, SciTe della distribuzione ufficiale non è utilizzabile.

\subject{Compilare Scite}
Per avere un SciTe che funzioni con {\Context} ho ricompilato, sotto windows, SciTe dai sorgenti.
Ho usato Mysys 2 (https://www.msys2.org/) e per i pacchetti vedi: \\
https://packages.msys2.org/package/mingw-w64-ucrt-x86_64-gtk2?repo=ucrt64 \\
Nel terminale \type{C:\msys64\msys2.exe} ho installato tutti i pacchetti per sviluppare in C/C++.
Ho usato la categoria mingw-w64-ucrt che supporta solo  Window10/11
(vedi https://github.com/libjxl/libjxl/blob/main/doc/developing_in_windows_msys.md per una spiegazione delle varie opzioni).
Questi sono i comandi che ho usato per installare il compilatore:
\startpowsh
pacman -S mingw-w64-ucrt-x86_64-gcc
pacman -S mingw-w64-ucrt-x86_64-gtk2
pacman -S mingw-w64-ucrt-x86_64-make
pacman -S --needed base-devel mingw-w64-ucrt-x86_64-toolchain
pacman -S git mingw-w64-ucrt-x86_64-cmake mingw-w64-ucrt-x86_64-ninja
pacman -S mingw-w64-ucrt-x86_64-gtest mingw-w64-ucrt-x86_64-giflib
pacman -S mingw-w64-ucrt-x86_64-libpng mingw-w64-ucrt-x86_64-libjpeg-turbo
cd /d/scratch/latex/ConTex/editor-scite/sourcecode-scite531/scintilla/win32
mingw32-make
\stoppowsh

SciTe si può compilare, sotto windows, solo nella versione win32, quindi i pacchetti Gtk si possono trascurare.

Siccome in Mysys2 ho diversi ambienti di compilazione installati è risultato che nella bash non ho più nessun gcc e mingw32-make.
Per questo ho settato nello script iniziale della bash, il file  \type{.bash_profile}, questo path:
\startpowsh
export PATH="$PATH:/c/msys64/ucrt64/bin"
\stoppowsh

Per la compilazione basta seguire il file Readme incluso nei sorgenti di SciTe, che poi è lanciare il programma mingw32-make.

\subsubject[lpegscite]{Lpeg in SciTe}

Da quanto ho capito, Lpeg è un modulo di Lua che non è integrato nei sorgenti nativi di SciTe. La spiegazione si trova
in un commento nel file \type{scite-context-lexer.lua}. L'autore dello script ha mantenuto una dll sulla falsa riga di ScintillaLua
per avere un Lexer di {\Context}. Ad un certo punto si è stufato di mantenere la dll con tutti i suoi cambiamenti di
interfaccia ed ha pensato bene di integrare Lpeg in Scite direttamente. A me interessa il codice sorgente di questo nuovo SciTe,
ma non sono riuscito a trovarlo.
Ho fatto dei tentativi di integrare Lpeg in Scite usando il compilatore mingw-w64 e
Lpeg scaricato dal suo sito  www.inf.puc-rio.br/$\sim$roberto/lpeg.
Inizialmente senza risultato in quanto non conosco bene i meccanismi di integrazione di Lua in SciTe.
Il meccanismo, però, è molto semplice. Basta compilare la libreria lpeg in SciTe e mettere la libreria Lpeg a disposizione
come, per esempio, viene fatto per la libreria Lua di string. Nella repository dove si trova questo file ho messo la patch usata.
Alla fine uso il file SciTe statico da me compilato:
\startpowsh
ConTex\scite\bin\SciTe-WithLpeg.exe
\stoppowsh
che ingloba  tutte le componenti necessarie, vale a dire scintilla, lexilla,  la stdlibc++ di mingw64 e Lpeg.
Il programma funziona molto bene con il Lexer  di {\ConTeXt},
mostra un ottima formattazione dei  file in formato .tex, esegue la generazione del file Pdf e mostra subito il file così creato
(es. con CTRL + F7).

\placefigure
    [][fig:ambientescite]
    {SciTe e Pdf.}
    {\externalfigure[./images/20-11-_2022_21-31-25.png][width=0.9\makeupwidth]}


\stoptext

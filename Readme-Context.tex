\setupbodyfont[pagella]

\definetyping[powsh][bodyfont=small]
\definetyping[sh][bodyfont=small]

\starttext

\title{ConText}
Vediamo di mettere un {\em Readme} usando questa volta, invece del file .md il formato .tex per avere un pdf.
\ConTeXt\ è un altro engine per creare Pdf partendo da un sorgente in Tex. È uno degli ultimi sviluppi per quanto
riguarda il mondo Tex e per me vale la pena di provare.

\subject{Installazione}
L'installazione non l'ho trovata immediata. C'è un Wiki che si trova su https://wiki.contextgarden.net/Installation
dove viene indicato un link per win64. Questo link, però, non riesco a scaricarlo in quanto lo zip sembra bloccato da Brave.
Ho allora dovuto usare in un WSL:
\startsh
wget http://lmtx.pragma-ade.com/install-lmtx/context-win64.zip
\stopsh

Ho anche scaricato predentemente uno zip che, da quanto ho capito, mi sembra essere il sorgente di tutto il progetto, quindi niente di
eseguibile. Questo zip l'ho scaricato da http://www.pragma-ade.com/download-1.htm e l'ho scompattato in

\type{D:\scratch\latex\ConTex}

Lo zip context-win64.zip per windows è anch'esso scompattato in:

\type{D:\scratch\latex\ConTex}.
In context-win64.zip si trova un installer che non fa altro che scaricare tutti gli eseguibili mancanti per win64, assieme a tutti i vari font, e
li mette nella sottodirectory tex. L'eseguibile è ora in

\type{D:\scratch\latex\ConTex\tex\texmf-win64\bin}

Ho creato la directory {\bold _my-project} dove ho messo i miei primi esperimenti con ConTex come questo file e altri esempi
introduttivi presi dal wiki o dal file install.pdf. La prima impressione non è affatto male.

\subject{Uso}
Per farli andare si usa il comando
\startpowsh
context Readme-Context.tex
\stoppowsh

così viene generato il file Readme-Context.pdf.

Nota bene che il comando va lanciato da cmd se si usa setpath.bat

In powershell è invece:
\startpowsh
$env:path = "D:\scratch\latex\ConTex\tex\texmf-win64\bin;" + $env:path
\stoppowsh

\subject{Scite}
Il programma SciTe è molto semplice da installare (non ha l'installer) e si trova su: https://www.scintilla.org/SciTEDownload.html.
Ho installato la versione a 64 bit. Per lanciarlo in combinazione con {\Context} si usa la command line
in \type{D:\scratch\latex\ConTex} usando il comando:
\startpowsh
setpath.bat
cd .\editor-scite\wscite
scite
\stoppowsh

Per l'installazione del lexer di  {\Context} in SciTe si va a copiare la directory
context in \type{D:\scratch\latex\ConTex\tex\texmf-context\context\data\scite} nella directory dove risiede SciTe.
Poi bisogna editare il file SciTEGlobal.properties aggiungendo la linea alla fine del file:

\type{import context/scite-context-user}

Per quanto riguarda il Lexer di {\Context}  in SciTe non riesco a farlo andare in quanto il modulo di Lua lpeg non viene
trovato.

SciTe usa come pdf viewer https://www.sumatrapdfreader.org/downloadafter che è un exe standalone.
L'ho messo nella stessa directory dove è {\Context}. Con CTRL + F7 si ha il pdf.

\subject{Compilare Scite}
Ho usato Mysys 2 (https://www.msys2.org/) per i pacchetti vedi: https://packages.msys2.org/package/mingw-w64-ucrt-x86_64-gtk2?repo=ucrt64 \\
Lancio il terminale \type{C:\msys64\msys2.exe} per poi installare tutti i pacchetti per sviluppare in C/C++.

\startpowsh
pacman -S mingw-w64-ucrt-x86_64-gcc
pacman -S mingw-w64-ucrt-x86_64-gtk2
pacman -S mingw-w64-ucrt-x86_64-make
pacman -S --needed base-devel mingw-w64-ucrt-x86_64-toolchain
pacman -S git mingw-w64-ucrt-x86_64-cmake mingw-w64-ucrt-x86_64-ninja
pacman -S mingw-w64-ucrt-x86_64-gtest mingw-w64-ucrt-x86_64-giflib
pacman -S mingw-w64-ucrt-x86_64-libpng mingw-w64-ucrt-x86_64-libjpeg-turbo
cd /d/scratch/latex/ConTex/editor-scite/sourcecode-scite531/scintilla/gtk
mingw32-make

\stoppowsh
Nota che uso i pacchetti ucrt che sono quelli compatibili con windows 10
(vedi https://github.com/libjxl/libjxl/blob/main/doc/developing_in_windows_msys.md per una spiegazione delle varie opzioni )

SciTe si può compilare solo nella versione win32.

\subsubject{Lpeg in SciTe}
Da quanto ho capito Lpeg è un modulo di LUA che non è integrato in SciTe. La spiegazione si trova
in un commento nel file scite-context-lexer.lua. L'autore aveva mantenuto una dll, sulla falsa riga di ScintillaLua,
per avere un Lexer di {\Context}. Ad un certo punto si è stufato di mantenere la dll con tutti i suoi cambiamenti di
interfaccia ed ha pensato bene di integrare Lpeg in Scite direttamente. A me interessa il codice sorgente di questo nuovo SciTe,
ma non sono riuscito a trovarlo. Ho fatto dei tentativi di integrare LPeg in Scite usando il compilatore mingw-w64 e
scaricando lpeg da http://www.inf.puc-rio.br/~roberto/lpeg ma ancora senza risultato, in quanto non conosco bene i
meccanismi di integrazione di Lua in SciTe.



\stoptext
